 \section{Estado del Arte  }
 En los últimos años, se han llevado a cabo varios proyectos e investigaciones para desarrollar sistemas que puedan detectar, medir y analizar precursores sísmicos, en esta sección, se presentan trabajos relacionados con el prototipo mencionado anteriormente, ya sea por el tema o por las tecnologías seleccionadas e implementadas en cada uno de ellos. Se elabora una tabla que incluye los aspectos relevantes que permiten respaldar la solución propuesta, seguida de una explicación y, finalmente, se muestran los proyectos relacionados que están actualmente en desarrollo. Algunos de los trabajos destacados en el campo incluyen:
 \subsection{TELESISMUX: Subsistema Telemático para una Estación Sísmica Remota }
 Este proyecto terminal presenta la investigación, diseño, desarrollo e implementación de un sistema telemático capaz de capturar, almacenar y transmitir información relativa a variables sísmicas medidas en una estación electrosísmica. El sistema integra un enlace de comunicación punto a punto entre un nodo en la estación sísmica y otro en el Laboratorio de Sistemas Complejos (UPIITA-IPN). También incluye un software de control y monitorización de los sistemas presentes en la estación sísmica y una aplicación web de acceso remoto. Este trabajo aborda los retos a los que se enfrentan los equipos de investigación en la recolección y transmisión de datos sísmicos para su posterior análisis. ~\cite{Valencia2012}
\subsection{Sistema IoT para la medición, registro y visualización web de señales electromagnéticas, geo-eléctricas y acústicas asociadas a precursores sísmicos}
Este proyecto propone el desarrollo de un sistema basado en IoT que pueda medir, registrar, almacenar y mostrar gráficos de señales electromagnéticas, geo-eléctricas y acústicas asociadas a precursores sísmicos. Los sensores y electrodos se colocan en regiones de alta actividad sísmica, y se implementa una interfaz web con servicios de almacenamiento en la nube. El sistema emplea el método VAN para medir señales electrosísmicas en la banda ULF del espectro electromagnético, diseña un sistema de inducción electromagnética para la banda VLF y registra señales acústicas terrestres. ~\cite{PerezS2022}
\subsection{Sistema de monitoreo y registro de señales asociadas a precursores sísmicos en la costa mexicana del Pacífico }
Un sistema que monitorea y registra diferentes señales relacionadas con precursores sísmicos, incluyendo señales geoeléctricas, electromagnéticas y variables ambientales como la humedad relativa y la conductividad eléctrica del suelo.~\cite{Guerrero2019}

\subsection{Anomalías en las señales electromagnéticas de baja frecuencia en la banda VLF}
Un estudio que examina las anomalías en las señales electromagnéticas de baja frecuencia en la banda VLF y su relación con los sismos de gran magnitud, así como el diseño de un sistema para monitorear y registrar señales de la banda VLF y las señales geomagnéticas terrestres.~\cite{DeLeon2020}

\begin{table}[h!]
\centering
\begin{tabular}{|p{4cm}|p{8cm}|p{4cm}|}
\hline
\textbf{Proyecto} & \textbf{Características principales} & \textbf{Contribución al trabajo actual} \\ \hline
TELESISMUX & 
\begin{itemize}
  \item Sistema telemático para estaciones sísmicas remotas
  \item Captura, almacena y transmite información relacionada con variables sísmicas
\end{itemize} & 
Base para el desarrollo del sistema de adquisición y transmisión de datos \\ \hline
Sistema IoT para la medición, registro y visualización web de señales electromagnéticas, geoléctricas y acústicas asociadas a precursores sísmicos & 
\begin{itemize}
  \item Sistema IoT integral
  \item Medición, registro y visualización de señales electromagnéticas, geoléctricas y acústicas
  \item Sensores y electrodos en regiones de alta actividad sísmica
\end{itemize} &
Registro y visualización de precursores sísmicos \\ \hline
Sistema de monitoreo y registro de señales asociadas a precursores sísmicos en la costa mexicana del Pacífico & 
\begin{itemize}
  \item Monitoreo y registro de señales geoeléctricas, electromagnéticas y variables ambientales
  \item Análisis de la relación entre estas señales y eventos sísmicos
\end{itemize} &
Incorporación de análisis de variables ambientales en el estudio de precursores sísmicos \\ \hline
Anomalías en las señales electromagnéticas de baja frecuencia en la banda VLF & 
\begin{itemize}
  \item Estudio de anomalías en señales electromagnéticas de baja frecuencia en la banda VLF
  \item Diseño de sistemas para monitorear y registrar señales de la banda VLF y señales geomagnéticas terrestres
  \item Análisis estadístico y de correlaciones temporales de las series de tiempo obtenidas
\end{itemize} &
Base para el análisis de señales electromagnéticas y su relación con eventos sísmicos \\ \hline
\end{tabular}
\caption{Características principales de los proyectos relacionados y su contribución al trabajo actual}
\label{tab:proyectos_relacionados}
\end{table}
\clearpage