\section{Planteamiento del problema}
En el campo de la sismología y la gestión del riesgo sísmico, existe una brecha significativa en el manejo y análisis de los datos recopilados de los posibles precursores sísmicos de diversas fuentes.Los sistemas actuales no son adecuados para analizar eficientemente estas mediciones, lo que dificulta obtener información precisa sobre la ocurrencia de un sismo en un período de tiempo específico. Esta deficiencia incide en la comprensión de los procesos que desencadenan los sismos.

La pregunta de investigación que se propone abordar en este proyecto terminal es: ¿Cómo se puede construir un sistema eficaz, integral y de fácil acceso para el procesamiento, análisis y visualización de datos de posibles precursores sísmicos provenientes de diferentes fuentes, que a su vez potencie la comprensión de los procesos sísmicos?

El proyecto que se presenta busca superar las limitaciones actuales en el manejo y análisis de datos de precursores sísmicos, desarrollando y aplicando un sistema unificado que pueda procesar eficientemente datos de diversas fuentes. Al centrarse en esta cuestión de investigación, el proyecto pretende contribuir a una mejor comprensión de los procesos que desencadenan los sismos, permitiendo en última instancia el desarrollo de estrategias de prevención y mitigación más sólidas.

Para lograr este objetivo, el proyecto implicará un enfoque sistemático y polifacético que incluye la identificación y selección de datos relevantes que estiman los posibles precursores sísmicos, el uso de algoritmos para el procesamiento y análisis de datos, la implementación de herramientas de graficación accesibles y fáciles de usar, y la evaluación de la eficacia del sistema en el análisis de los procesos sísmicos.