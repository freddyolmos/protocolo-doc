
\section{Marco teórico}
Un sismo es un movimiento brusco de la corteza terrestre, provocado por la liberación de energía acumulada en forma de ondas sísmicas. Debido a la naturaleza no lineal y dinámica de los sismos, actualmente es imposible predecir con precisión dónde ocurrirán, cuál será su magnitud y el daño que podrían causar.

La República Mexicana se encuentra en una de las regiones sísmicas más activas del mundo, conocida como el Cinturón Circumpacífico. Esta alta actividad sísmica se debe a la interacción de varias placas tectónicas, incluyendo Norteamérica, Cocos, Pacífico, Rivera y Caribe, así como a fallas locales en varios estados. Los estados más afectados son Chiapas, Guerrero, Oaxaca, Michoacán, Colima y Jalisco, debido a la interacción de las placas oceánicas de Cocos y Rivera con Norteamérica y el Caribe en la costa del Pacífico. Aunque los epicentros se localizan en diversos puntos del Pacífico, la Ciudad de México, aunque no está en la costa, se ve afectada debido a su proximidad y a la naturaleza de su terreno. El estudio de la actividad sísmica en México comenzó a principios del siglo XX, y se inauguró la red sismológica mexicana en 1910. Actualmente, el Servicio Sismológico Nacional \cite{SSN} opera una red de 35 estaciones sismológicas y reporta en promedio la ocurrencia de 4 sismos por día de magnitud M > 3.0. Además del SSN, existen otros grupos de investigación como el CICESE y la RESNOR que estudian la actividad sísmica en el Golfo de California y la falla de San Andrés, respectivamente. También hay instituciones educativas que realizan estudios de sismicidad regional y mantienen comunicación para compartir avances.

En las últimas décadas se han realizado numerosos estudios para identificar posibles precursores de sismos, que son fenómenos observables que preceden a los fenómenos sísmicos. Estos precursores abarcan los patrones de quietud sísmica \cite{SSN}, las anomalías en los campos eléctricos y magnéticos, las emisiones de gases/aerosoles, las señales ionosféricas, los cambios en el nivel de las aguas subterráneas, las variaciones de temperatura de la superficie, las deformaciones de la superficie, el comportamiento de los animales, las señales térmicas infrarrojas, las ondas gravitacionales atmosféricas y los relámpagos. 

La investigación sistemática de los precursores de sismos comenzó en Japón en la década de 1960 y desde entonces se ha ampliado mediante colaboraciones internacionales identificando anomalías en la medición de diversas señales previas a la manifestación de un sismo de gran magnitud (Ms >=6) como Schall 1988, Varotsos 1993, McGuire 1994, Kagan 2008.
A pesar de estas observaciones, la predicción sísmica sigue siendo un desafío futuro. Aunque todavía no se ha logrado un éxito concreto, el estudio de fenómenos potencialmente asociados a la ocurrencia de movimientos telúricos ha avanzado de manera progresiva, desarrollando y expandiendo nuevas redes de monitoreo en diversas áreas con alta actividad sísmica alrededor del mundo~\cite{Bhardwaj2021}.
\subsection{Posibles precursores sismicos}
\subsubsection{Patrones de quietud sísmica }
La quietud sísmica es un fenómeno caracterizado por una notable reducción de la actividad sísmica en una región concreta, que según algunos expertos podría servir de precursor de un gran (Ms >=6) sismo. El concepto subyacente es que una disminución de los eventos sísmicos menores puede preceder un sismo de gran magnitud (Ms >=6), ya que la tensión se acumula a lo largo de las fallas antes de desencadenarse finalmente en un evento sísmico significativo~\cite{SGM}~\cite{mcnally1983seismic}~\cite{wyss1997cannot} \cite{scholz1997whatever}.
%pag 33
\subsubsection{Anomalías del campo eléctrico }
Las anomalías del campo eléctrico se refieren a variaciones o perturbaciones inusuales en el campo eléctrico de la Tierra, que han sido observadas antes de algunos sismos. Algunos investigadores sugieren que estas anomalías podrían deberse a cambios inducidos por tensiones en la corteza terrestre, que provocan la generación de cargas eléctricas y alteraciones en el campo eléctrico terrestre~\cite{varotsos1984physical}~\cite{yepez1995electric} se han estudiado en zonas como Grecia y Japon.
\subsubsection{Anomalías del campo magnético}
Las anomalías del campo magnético se refieren a cambios o variaciones inesperados en el campo magnético de la Tierra que se han observado en asociación con algunos sismos. Estas anomalías podrían estar relacionadas con cambios inducidos por tensiones en la corteza terrestre, que pueden alterar las propiedades magnéticas de las rocas y producir cambios en el campo magnético ~\cite{hayakawa1999fractal}~\cite{hayakawa2007monitoring}.
\subsubsection{Señales ionosféricas}
La ionosfera es una región débilmente ionizada de la atmósfera terrestre, situada por encima de los 70-80 km de altitud donde los electrones libres y los iones forman un plasma. La ionización se produce principalmente por la radiación solar en el rango de los rayos X y el ultravioleta extremo. La densidad de electrones a distintas altitudes permite diferenciar cuatro regiones dentro de la ionosfera, cada una de las cuales posee densidades, alturas y frecuencias máximas de plasma distintas. Debido a su naturaleza no homogénea, la ionosfera puede ejercer efectos variables sobre las ondas electromagnéticas que viajan en su interior, lo que puede influir en la detección de anomalías asociadas a los sismos~\cite{eftaxias2003experience}.

La ionosfera terrestre contiene una elevada concentración de iones y electrones libres que pueden interactuar con las ondas electromagnéticas. Algunos estudios han informado de cambios o perturbaciones inusuales en la ionosfera que preceden a ciertos sismos, como fluctuaciones en el contenido total de electrones (TEC) o alteraciones en la propagación de las ondas de radio.

El acoplamiento litosfera-atmósfera-ionosfera describe la conexión física entre los fenómenos observados en la ionosfera y los seísmos que parecen desencadenarlos. Existen numerosas teorías en función del tipo de fenómeno considerado, pero la teoría más completa ha sido elaborada por Pulinets y Ouzounov. Las anomalías de conductividad creadas en la superficie terrestre se transfieren a la ionosfera a través del Circuito Eléctrico Global, un sistema cuasi estacionario de corrientes eléctricas entre la superficie y la ionosfera. La primera observación del impacto de un sismo en la ionosfera se produjo en 1964. Al principio, los investigadores se centraron en las anomalías observadas en los parámetros de las distintas capas, detectadas por ionosondas y radares, o in situ por satélites.
\subsection{Elementos clave para la predicción de sismos }
Allen propuso seis elementos clave que deben tenerse en cuenta en la predicción de sismos.~\cite{Bhardwaj2021} Estos elementos son 

  
\begin{itemize}
    \item Ventana temporal \\
La ventana temporal se refiere al periodo de tiempo dentro del cual se espera que se produzca un sismo. La estimación precisa de la ventana temporal es esencial para una preparación y respuesta eficaces ante las catástrofes. 

    \item Ventana espacial \\
La ventana espacial indica la ubicación geográfica en la que se prevé que se produzca un sismo. La determinación precisa de la ventana espacial es crucial para orientar los esfuerzos de mitigación y respuesta. 

    \item Ventana de magnitud \\
La ventana de magnitud representa la intensidad prevista del sismo. La estimación precisa de la ventana de magnitud es importante para comprender el impacto potencial y planificar las contramedidas adecuadas. 

    \item Grado de confianza \\
El grado de confianza se refiere a la certeza con la que se realiza la predicción. Las predicciones de alto grado de confianza pueden ayudar a los responsables de la toma de decisiones a adoptar las medidas preventivas adecuadas y a asignar los recursos de forma eficaz. 

\item Sismos imprevisibles como sucesos aleatorios \\
La posibilidad de que se produzcan seísmos imprevisibles como sucesos aleatorios debe tenerse en cuenta a la hora de hacer predicciones, ya que puede afectar significativamente a la fiabilidad global de la previsión. 

\item Documentación accesible y comprensible \\
Una documentación fácilmente accesible y comprensible es fundamental para garantizar la eficacia de las evaluaciones futuras y facilitar la toma de decisiones con conocimiento de causa. 
\end{itemize}

La incorporación de estos elementos clave a la investigación sobre el analisis de sismos es esencial para garantizar que la información proporcionada sea precisa, fiable y útil para adoptar las medidas preventivas adecuadas. 
\clearpage