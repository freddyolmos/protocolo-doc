\section{Objetivo general }
Desarrollar un sistema unificado para la recolección, análisis y visualización de datos de múltiples fuentes relacionados con posibles precursores sísmicos, centrándose en patrones de quietud sísmica, anomalías del campo eléctrico, anomalías del campo magnético y señales ionosféricas, con el objetivo de mejorar la comprensión de ocurrencia de sismos en relación a sus posibles precursores.

\subsection{Objetivos específicos}
\begin{itemize}
    \item Recolección de datos de eventos sismicos, anomalías del campo eléctrico, anomalías del campo magnético, señales ionosféricas, y datos secundarios procedentes de diversas fuentes 

    \item Procesamiento de datos recolectados aplicando algoritmos para el procesamiento y análisis de los datos de los posibles precursores sísmicos. 
    
    \item Utilización de modelos matemáticos ya conocidos que incorporen los cálculos de posibles precursores sísmicos identificados para establecer patrones de comportamiento relevantes de dichos precursores.
    
    \item Creación de un sistema de visualización para mostrar los datos procesados de los posibles precursores sísmicos.

\end{itemize}
\clearpage