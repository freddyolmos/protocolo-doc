\section{Escenario de pruebas}
El escenario de pruebas se llevara a cabo con datos registrados o medidos en diferentes épocas en las que las estaciones de monitoreo han estado en funcionamiento. Los pasos a seguir son los siguientes:

\textit{Autenticación de usuario:} El usuario inicia sesión y accede al panel de control del sistema.

\textit{Elección de precursor a analizar:} Se selecciona el posible precursor sísmico como anomalías del campo eléctrico, magnético y señales ionosféricas de los datos que ya se cuenta.

\textit{Selección de región/fecha:} El usuario elige una región de interés en un mapa interactivo (reservado a locaciones con datos de estaciones de monitoreo) así como la fecha mediante o menu desplegable de ventanas de tiempo válidas.

\textit{Análisis y estimación:} El sistema ejecuta las rutinas que cada precursor requiere para realizar sus estimaciones.

\textit{Visualización de resultados:} El sistema muestra gráficos, tablas y mapas para que los investigadores realicen las interpretaciones correspondientes.

\textit{Exportación de resultados:} El usuario exporta los resultados en formatos PDF, CSV o imágenes segun corresponda.

\textit{Cierre de sesión:} El usuario cierra sesión tras revisar los resultados.

Este escenario de prueba evalúa la funcionalidad, usabilidad, precisión y efectividad del sistema de análisis de precursores sísmicos y estimación de probabilidad de ocurrencia de sismo
\clearpage