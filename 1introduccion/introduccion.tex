\section{Introducción}
La República Mexicana se encuentra ubicada en una de las regiones sísmicamente más activas del mundo, conocida como el Cinturón Circumpacífico. Esta región concentra la mayor actividad sísmica del planeta, lo que convierte a México en un país propenso a los sismos. La interacción de varias placas tectónicas, como Norteamérica, Cocos, Pacífico, Rivera y Caribe, es la principal causa de la alta sismicidad en el país. Los estados más afectados por esta actividad son Chiapas, Guerrero, Oaxaca, Michoacán, Colima y Jalisco, debido a la interacción de las placas oceánicas de Cocos y Rivera con Norteamérica y el Caribe en la costa del Pacífico\cite{SGM2023}.

Aunque la Ciudad de México no se encuentra en la costa, su cercanía a las zonas sísmicas la expone a los efectos sísmicos. Además, la naturaleza de su terreno, que alguna vez fue un lago, contribuye a la preocupación por la vulnerabilidad sísmica de la ciudad. A lo largo de los años, se ha observado y registrado la actividad sísmica en México, con el Servicio Sismológico Nacional encargado de recopilar y analizar estos datos. Existen también otros grupos de investigación y redes que trabajan en el estudio de la sismicidad en diferentes partes del país.

En la actualidad, el riesgo sísmico se ha convertido en una preocupación cada vez mayor en todo el mundo, especialmente en áreas propensas a los sismos como México. La prevención y mitigación de los efectos devastadores de los sismos en la población y la infraestructura son fundamentales para la gestión del riesgo de desastres. Sin embargo, la comprensión de los precursores sísmicos y la toma de decisiones informadas en la gestión del riesgo sísmico se ven obstaculizadas por la limitada disponibilidad de datos y la falta de un enfoque sistemático para su análisis.

Este documento presenta un proyecto que tiene como objetivo abordar esta brecha. Se propone el desarrollo de un "Sistema unificado para recolectar, analizar y graficar datos de múltiples fuentes relacionados con posibles precursores sísmicos". Este sistema integrado y accesible permitirá procesar, analizar y visualizar datos de posibles precursores sísmicos provenientes de diversas fuentes, como patrones de quietud sísmica, anomalías del campo eléctrico, anomalías del campo magnético y señales ionosféricas. Para mejorar la comprensión de la ocurrencia de sismos en relacion con los precursores antes mencionados. 

En las siguientes secciones, se profundizará en la justificación del proyecto, el planteamiento del problema, la propuesta de solución y los objetivos generales y específicos. Además, se describirán las etapas del proyecto, que incluirán la recolección, el almacenamiento, el procesamiento y análisis de datos, y la visualización de la información obtenida. Se espera que esta propuesta contribuya a una mejor comprensión de los procesos que desencadenan los sismos, lo cual permitirá proporcionar información para  el desarrollo de estrategias de prevención y mitigación más sólidas y eficaces en México.