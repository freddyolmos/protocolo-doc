\section{Introducción}
El cubo Rubik, creado en 1974 por el arquitecto húngaro Ernő Rubik como herramienta didáctica para mostrar transformaciones espaciales, se ha consolidado como uno de los rompecabezas mas populares del mundo, con más de 350 millones de unidades hasta 2018\cite{Reese2020}. Su impacto trascendió el entretenimiento, dando origen en los años 80 a competencias formales centradas en resolver el cubo en el menor tiempo posible. Esta disciplina, conocida como \textit{speed-cubing}, ha evolucionado hacia niveles altamente competitivos y técnicos.

Dentro de esta comunidad, el método Fridich -también denominado CFOP por sus siglas en inglés: Cross, F2L, OLL y PLL- se ha convertido en el sistema de resolución más utilizado por los practicantes de alto nivel\cite{Boyce2022b}. A diferencia del método básico para principiantes, CFOP requiere la memorización y ejecución precisa de más de 78 algoritmos. Particularmente, la etapa F2L (First Two Layers) representa el mayor desafio, ya que concentra aproximadamente el 52\% del tiempo de resolución total y genera la mayoria de las pausas que interrumpen la fluidez del solve\cite{Boyce2022b}.

Sin embargo, el proceso de aprendizaje de estos algoritmos es considerablemente ineficiente. Para practicar un caso específico, el usuario debe manipular el cubo de forma manual hasta reproducir la situación deseada, resolviéndolo o desarmandolo repetidamente. Esta limitación introduce una barrera mecénica que dificulta la práctica enfocada, ralentiza la memorización y puede llevar a la frustración del practicante.

Frente a este problema, la ingeniería mecatrónica ofrece una solución prometedora. Esta disciplina, entendida como la integración sinérgica de sistemas mecánicos, electrónicos, informáticos y de control, permite el diseño de sistemas inteligentes que interactúan con el entorno de forma precisa y programada\cite{Perdomo2003}. Aplicada al contexto del \textit{speed-cubing}, esta integración da lugar aa un sistema robotizado capaz de automatizar la preparación de casos específicos del método Fridrich, eliminando la necesidad de intervención manual.

Si bien existen numerosos proyectos de robótica enfocados en resolver completamente el cubo Rubik -por ejemplo, sistemas que emplean visión artificial y brazos colaborativos para identificar y ejecutar soluciones completas en pocos segundos\cite{Gorriz2023}-, la mayoria de estos desarrollos se centran en la resolución total del cubo y no están pensados como herramientas de entrenamiento. Su propósito final es resolver, no generar situaciones controladas para praticar partes específicas del proceso. 

La literatura reporta numerosos robots cuyo objetivo principal es resolver completamente el cubo (p.ej., sistema UR3e + OpenCV que detecta el estado y ejecuta la solución con un manipulador colaborativo\cite{Gorriz2023}). Sin embargo, dichos trabajitos no ofrecen funciones de entrenamiento controlador caso-por-caso; tras cada solución el cubo vuelve a estado resuelto, obligando al practicante a iniciar de cero. Esto evidencia un vacío que el presente proyecto aborda al enfocarse en la preparación selectiva de F2L, OLL y PLL.

Se propone un robot de sobremesa con actuadores en las seis caras del cubo, controlado desde una interfaz de escritorio. El usuario elige un caso (p.ej., OLL) y el sistema aplica el algoritmo inverso para dejar el cubo listo en manos de X s. Un botón "Repetir" regenera la misma situación indefinidamente; un modo "Random" genera casos aleatorios para evaluación. En versiones futuras se integrará vision artificial para verificar la ejecución y ofrecer feedback automático.

Este proyecto tiene el potencial de impactar positivamente en distintos sectores. En primer lugar, para los speed-cubers, el sistema representa una herramienta de gran valor, ya que permite reducir significativamente el tiempo invertido en la preparación de casos específicos del método Fridrich. Al automatizar este proceso, los usuarios pueden centrarse directamente en la memorización y ejecución del algoritmo, acelerando su curva de aprendizaje.

Por otro lado, el robot también tiene aplicaciones dentro del ámbito educativo, particularmente en instituciones enfocadas en áreas STEM (Ciencia, Tecnología, Ingeniería y Matemáticas). Al combinar mecánica, electrónica, control e interfaces digitales en un solo sistema funcional, el proyecto se convierte en un recurso tangible y atractivo para enseñar conceptos complejos de manera interactiva y práctica. 

Además, desde el enfoque de la investigación, este tipo de plataforma ofrece una base interesante para estudiar temas como el aprendizaje motor, la repetición y memorización de patrones, así como para experimentar con diferentes estrategias de optimización en la resolución de algoritmos. Su estructura abierta permitiría a otros investigadores extender o adaptar el sistema a nuevas funciones. 

A partir del análisis de antecedentes y la identificación de una necesidad no resuelta en el ámbito del aprendizaje avanzado del cubo Rubik, surge la motivación para desarrollar este proyecto. El hecho de que no existan actualmente sistemas que permitan generar, de forma controlada y repetitiva, casos específicos del método Fridrich, representa una oportunidad clara de innovación.

En este contexto, el objetivo general del proyecto es diseñar, construir y validar un sistema mecatrónico que automatice la generación de casos CFOP, facilitando asi su práctica de manera eficiente. Para lograrlo, será necesario comprender a fondo la cinemática del cubo y traducirla en movimientos precisos ejecutados por un mecanismo robótico, A la para, se desarrollará una interfaz de usuario intuitiva que permita seleccionar casos, controlar el sistema y visualizar los algoritmos implicados. Finalmente, se evaluará la efectividad del robot mediante métricas como el tiempo requerido para generar cada caso, la precisión en los movimientos realizados y la repetibilidad del proceso, con el fin de validar su viabilidad como herramienta de entrenamiento. 