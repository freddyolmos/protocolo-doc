\section{Introducción}
El cubo Rubik, creado en 1974 por el arquitecto húngaro Ernő Rubik como ayuda docente para mostrar transformaciones espaciales, se convirtió en el juguete rompecabezas más vendido del mundo, con más de 350 millones de unidades hasta 2018\cite{SGM2023}. A partir de los años 80 surgieron competiciones oficiales, dando origen al speed-cubing, disciplina que premia el menor tiempo de resolución.

Aunque el método Fridrich (CFOP) es hoy "el sistema de resolucion mas utilizado" entre los competidores\cite{SGM2023}, dominarlo exige memorizar poco mas de 78 algoritmos y dedicar gran parte del tiempo a la etapa F2L, responsable del 52\% del tiempo total y de la mayoria de las pausas\cite{SGM2023}. La práctica manual resulta tediosa poruqe el alumno debe generar cada caso resolviendo o dearmando el cubo una y otra vez, lo que ralentiza la memorización y provoca frustración.

La mecatrónica se define como la integración sinérgica de mecánica, electrónica, informática y control para crear sistemas digitales\cite{SGM2023}. Un robot entrenador de algoritmos CFOP ejemplifica esta convergencia: mecanismos precisos para girar el cubo, electrónica de potencia y sensórica para ejecutar movimientos repetibles y una aplicación de software que calcula inversiones de algoritmos y registra el prograso del usuario. Al automatizar la generaración de casos, el sistema elimina la barrera mecánica y permite centrar el aprendizaje en la cognición y la memoria procedimental. 

La literatura repotar numerosos robots cuyo objetivo principal es resolver completamente el cubo (p.ej., sistema UR3e + OpenCV que detecta el estado y ejecuta la solución con un manipulador colaborativo\cite{SGM2023}). Sin embargo, dichos trabajitos no ofrecen funciones de entrenamiento controlador caso-por-caso; tras cada solución el cubo vuelve a estado resuelto, obligando al practicante a iniciar de cero. Esto evidencia un vacío que el presente proyecto aborda al enfocarse en la preparación selectiva de F2L, OLL y PLL.

Se propone un robot de sobremesa con actuadores en las seis caras del cubo, controlado desde una interfaz de escritorio. El usuario elige un caso (p.ej., OLL) y el sistema aplica el algoritmo inverso para dejar el cubo listo en manos de X s. Un botón "Repetir" regenera la misma situación indefinidamente; un modo "Random" genera casos aleatorios para evaluación. En versiones futuras se integrará vision artificial para verificar la ejecución y ofrecer feedback automático.

Este proyecto tiene el potencial de impactar positivamente en distintos sectores. En primer lugar, para los speed-cubers, el sistema representa una herramienta de gran valor, ya que permite reducir significativamente el tiempo invertido en la preparación de casos específicos del método Fridrich. Al automatizar este proceso, los usuarios pueden centrarse directamente en la memorización y ejecución del algoritmo, acelerando su curva de aprendizaje.

Por otro lado, el robot también tiene aplicaciones dentro del ámbito educativo, particularmente en instituciones enfocadas en áreas STEM (Ciencia, Tecnología, Ingeniería y Matemáticas). Al combinar mecánica, electrónica, control e interfaces digitales en un solo sistema funcional, el proyecto se convierte en un recurso tangible y atractivo para enseñar conceptos complejos de manera interactiva y práctica. 

Además, desde el enfoque de la investigación, este tipo de plataforma ofrece una base interesante para estudiar temas como el aprendizaje motor, la repetición y memorización de patrones, así como para experimentar con diferentes estrategias de optimización en la resolución de algoritmos. Su estructura abierta permitiría a otros investigadores extender o adaptar el sistema a nuevas funciones. 

A partir del análisis de antecedentes y la identificación de una necesidad no resuelta en el ámbito del aprendizaje avanzado del cubo Rubik, surge la motivación para desarrollar este proyecto. El hecho de que no existan actualmente sistemas que permitan generar, de forma controlada y repetitiva, casos específicos del método Fridrich, representa una oportunidad clara de innovación.

En este contexto, el objetivo general del proyecto es diseñar, construir y validar un sistema mecatrónico que automatice la generación de casos CFOP, facilitando asi su práctica de manera eficiente. Para lograrlo, será necesario comprender a fondo la cinemática del cubo y traducirla en movimientos precisos ejecutados por un mecanismo robótico, A la para, se desarrollará una interfaz de usuario intuitiva que permita seleccionar casos, controlar el sistema y visualizar los algoritmos implicados. Finalmente, se evaluará la efectividad del robot mediante métricas como el tiempo requerido para generar cada caso, la precisión en los movimientos realizados y la repetibilidad del proceso, con el fin de validar su viabilidad como herramienta de entrenamiento. 