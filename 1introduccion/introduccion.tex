\section{Introducción}
El cubo Rubik, creado en 1974 por el arquitecto húngaro Ernő Rubik como herramienta didáctica para mostrar transformaciones espaciales, se ha consolidado como uno de los rompecabezas mas populares del mundo, con más de 350 millones de unidades hasta 2018\cite{Reese2020}. Su impacto trascendió el entretenimiento, dando origen en los años 80 a competencias formales centradas en resolver el cubo en el menor tiempo posible. Esta disciplina, conocida como \textit{speed-cubing}, ha evolucionado hacia niveles altamente competitivos y técnicos.

Dentro de esta comunidad, el método Fridich -también denominado CFOP por sus siglas en inglés: Cross, F2L, OLL y PLL- se ha convertido en el sistema de resolución más utilizado por los practicantes de alto nivel\cite{Boyce2022b}. A diferencia del método básico para principiantes, CFOP requiere la memorización y ejecución precisa de más de 78 algoritmos. Particularmente, la etapa F2L (First Two Layers) representa el mayor desafio, ya que concentra aproximadamente el 52\% del tiempo de resolución total y genera la mayoria de las pausas que interrumpen la fluidez del solve\cite{Boyce2022b}.

Sin embargo, el proceso de aprendizaje de estos algoritmos es considerablemente ineficiente. Para practicar un caso específico, el usuario debe manipular el cubo de forma manual hasta reproducir la situación deseada, resolviéndolo o desarmandolo repetidamente. Esta limitación introduce una barrera mecánica que dificulta la práctica enfocada, ralentiza la memorización y puede llevar a la frustración del practicante.

Frente a este problema, la ingeniería mecatrónica ofrece una solución prometedora. Esta disciplina, entendida como la integración sinérgica de sistemas mecánicos, electrónicos, informáticos y de control, permite el diseño de sistemas inteligentes que interactúan con el entorno de forma precisa y programada\cite{Perdomo2003}. Aplicada al contexto del \textit{speed-cubing}, esta integración da lugar a un sistema robotizado capaz de automatizar la preparación de casos específicos del método Fridrich, eliminando la necesidad de intervención manual.

Si bien existen numerosos proyectos de robótica enfocados en resolver completamente el cubo Rubik -por ejemplo, sistemas que emplean visión artificial y brazos colaborativos para identificar y ejecutar soluciones completas en pocos segundos\cite{Gorriz2023}-, la mayoria de estos desarrollos se centran en la resolución total del cubo y no están pensados como herramientas de entrenamiento. Su propósito final es resolver, no generar situaciones controladas para praticar partes específicas del proceso. 

En este sentido, el presente protocolo propone el diseño y construcción de un robot entrenador, con la capacidad de posicionar el cubo en configuraciones específicas de las etapas F2L, OLL o PLL del método CFOP. El sistema consistirá en un robot de sobremesa con actuadores que permitan girar las seis caras del cubo, controlado desde una aplicación de escritorio que proporcionará al usuario una interfaz visual para seleccionar el caso deseado. Mediante la aplicación del algoritmo inverso, el robot generará el estado inicial correspondiente para que el usuario pueda practicarlo de forma inmediata.

Este enfoque ofrece múltiples beneficios. Para los speed-cubers, representa una herramienta de alto valor, ya que reduce drásticamente el tiempo dedicado a preparar manualmente cada caso y permite concentrarse directamente en la ejecución del algoritmo. Desde el ámbito educativo, se convierte en un recurso tangible para instituciones enfocadas en áreas STEM, permitiendo enseñar de manera práctica temas como cinemática, electrónica, control y programación. Por último, en el campo de la investigación, esta plataforma abre la puerta al estudio de procesos de aprendizaje motor, memorización por repetición y diseño de sistemas interactivos aplicados a la enseñanza.

A partir de los antecedentes revisados, se identifica una necesidad no resuelta dentro del proceso de aprendizaje avanzado del cubo Rubik. Actualmente, no existen herramientas que permitan generar, de forma precisa y repetitiva, casos específicos del método Fridrich con fines didácticos. En respuesta a esta brecha, el presente proyecto se plantea como objetivo general diseñar, construir y validar un sistema mecatrónico que automatice la generación de dichos casos, facilitando su práctica de manera eficiente.

Para ello, se desarrollará un modelo cinemático que describa los movimientos del cubo, se diseñará el hardware robótico junto con su electrónica de control, se programará la interfaz de usuario con capacidad de inversión de algoritmos, y finalmente, se evaluará la precisión y repetibilidad del sistema a través de métricas como el tiempo de preparación de cada caso y la exactitud angular de los giros. Esta propuesta busca no solo resolver un problema técnico, sino también aportar una herramienta educativa y formativa con aplicaciones reales en el ámbito académico y deportivo.