\begin{titlepage}
         \begin{center}
            
              \vfill
              
            \begin{center}
                %begin{minipage}{0.2\textwidth}
                %    \begin{figure}[H]
                %    \includegraphics[height=36mm]{img/ipn.png}
                %    \end{figure}
                %\end{minipage}
                %\begin{minipage}{0.6\textwidth}
                    {\Large \bf Instituto Politécnico Nacional }
                    
                    \vfill
                    
                    Unidad Profesional Interdisciplinaria en \\
                    Ingeniería y Tecnologías Avanzadas
                    
                    \vfill
                %\end{minipage}
                % \begin{minipage}{0.2\textwidth}
                %    \begin{figure}[H]
                %    \includegraphics[height=36mm]{img/upiita-logo.png}
                %    \end{figure}
                %\end{minipage}
            \end{center}
            
              \vfill
        
        
              \textbf{\Large "Sistema mecatrónico de entrenamiento asistido para el aprendizaje del método Fridrich en el cubo Rubik 3 × 3"}\\[10pt]
            
              \vfill
            
              {\em  Alumno:} 
            
              \vfill
            
              {\large \bf Fidel Alfredo Olmos García} \\[10pt] 
                
              \vfill
          
          {\em  Asesores:} 

            \vfill


            \begin{center}
                \begin{tabular}{>{\raggedright\arraybackslash}p{3cm} p{10cm}}
                  Nombre:      & M. en C. Jorge Fonseca Campos\\
                  Procedencia: & UPIITA – IPN \\
                  Nombre:      & Dra. Irene Lopez Rodriguez \\
                  Procedencia: & Externa  \\
                    Nombre:      & Dr. Carlos Carrizales Velazquez \\
                  Procedencia: & Externo  \\
                \end{tabular}
            \end{center}

            \vfill
            \textbf{Junio, 2025}

 
        \end{center}
    \begin{abstract}
    El presente documento se enfoca en la necesidad de un sistema unificado para la recolección y procesamiento de datos de posibles precursores sísmicos, como patrones de quietud sísmica, anomalías en campos eléctricos y magnéticos, y señales ionosféricas, provenientes de diversas fuentes. El objetivo primordial de este proyecto es diseñar una herramienta integral y accesible para el análisis e identificación de estos datos, con el fin de mejorar la comprensión de los procesos sísmicos. Este proyecto está organizado en cuatro etapas esenciales: visualización de datos, procesamiento en un servidor dedicado, recolección de datos y almacenamiento de los mismos.
        \\ \\
        \textbf{Palabras clave:} precursores sísmicos, quietud sísmica, señales ionosféricas, campo magnético, campo eléctrico.
    \end{abstract}
        
    \end{titlepage}