\begin{titlepage}
         \begin{center}
            
              \vfill
              
            \begin{center}
                %begin{minipage}{0.2\textwidth}
                %    \begin{figure}[H]
                %    \includegraphics[height=36mm]{img/ipn.png}
                %    \end{figure}
                %\end{minipage}
                %\begin{minipage}{0.6\textwidth}
                    {\Large \bf Instituto Politécnico Nacional }
                    
                    \vfill
                    
                    Unidad Profesional Interdisciplinaria en \\
                    Ingeniería y Tecnologías Avanzadas
                    
                    \vfill
                %\end{minipage}
                % \begin{minipage}{0.2\textwidth}
                %    \begin{figure}[H]
                %    \includegraphics[height=36mm]{img/upiita-logo.png}
                %    \end{figure}
                %\end{minipage}
            \end{center}
            
              \vfill
        
        
              \textbf{\Large "Sistema unificado para recolectar, analizar y graficar datos de múltiples fuentes relacionados con posibles precursores sísmicos"}\\[10pt]

            \vfill
            \textbf{Mayo, 2023}

 
        \end{center}
    \begin{abstract}
       El presente documento aborda la necesidad de desarrollar un sistema unificado para la recopilación y procesamiento de datos de precursores sísmicos provenientes de diversas fuentes. Particularmente, se enfoca en la identificación de patrones de quietud sísmica \cite{SSN}, anomalías del campo eléctrico, anomalías del campo magnético y señales ionosféricas. El objetivo principal del proyecto es proporcionar una solución completa y accesible para el análisis e identificación de datos de precursores sísmicos, lo que permitirá el desarrollo de estrategias más sólidas de prevención y mitigación.

Las cuatro etapas fundamentales del proyecto incluyen una etapa de visualización de datos, un servidor en el que se procesarán los datos, una etapa de recolección de datos y una etapa de almacenamiento de datos. \\ \\
        \textbf{Palabras clave:} precursores sísmicos, quietud sismica, señales ionosféricas, campo magnetico, campo electrico 
    \end{abstract}
        
    \end{titlepage}