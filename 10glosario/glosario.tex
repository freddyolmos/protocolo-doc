\section{Glosario de términos}
\begin{itemize}

    \item Acoplamiento litosfera-atmósfera-ionosfera: Describe la conexión física entre los fenómenos observados en la ionosfera y los sismos que parecen desencadenarlos.

    \item Anomalías del campo eléctrico: Cambios inusuales en el campo eléctrico terrestre que podrían estar relacionados con la actividad sísmica.
    
    \item Anomalías del campo magnético: Cambios inusuales en el campo magnético terrestre que podrían estar relacionados con la actividad sísmica.
    
    \item Cinturón Circumpacífico: También conocido como el 'Cinturón de Fuego del Pacífico', es el área geográficamente más activa del mundo en términos de sismicidad y actividad volcánica.
    
    \item Contenido Total de Electrones (TEC): Medida de la cantidad total de electrones libres en la atmósfera terrestre entre dos puntos, a menudo usada en estudios de anomalías ionosféricas.
    
    \item Epicentro: Punto en la superficie de la Tierra directamente por encima del hipocentro o foco de un sismo.

    \item Grado de Confianza: Medida de la certeza con la que se realiza una predicción sísmica.
    
    \item Placas tectónicas: Grandes placas de roca sólida que componen la superficie de la Tierra, cuyos movimientos y interacciones pueden causar sismos.
    
    \item Precursor sísmico: Fenómeno observable que ocurre antes de un sismo y que podría indicar su inminencia.

    \item Señales ionosféricas: Variaciones en las propiedades de la ionosfera, como la densidad de electrones, que podrían estar relacionadas con la actividad sísmica.

    \item Servicio Sismológico Nacional (SSN): Organización encargada de registrar, analizar y reportar la actividad sísmica en México.

    \item Ventana Espacial: Ubicación geográfica en la que se prevé que se produzca un sismo en un modelo de predicción.

    \item Ventana de Magnitud: Rango de magnitudes dentro del cual se espera que caiga la magnitud de un sismo en un modelo de predicción.
    
    \item Ventana Temporal: Período de tiempo dentro del cual se espera que se produzca un sismo en un modelo de predicción.

    \item Web scraping: Técnicas y herramientas de programación utilizadas para extraer datos de manera automatizada desde páginas web.
    
\end{itemize}
